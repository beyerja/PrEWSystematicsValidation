\documentclass[xcolor=dvipsnames,compress,DESYcyan,9pt,unknownkeysallowed]{beamer}

\usepackage{blindtext}
% \usepackage[utf8,latin1]{inputenc}
\usepackage[T1]{fontenc}
\usepackage[english]{babel}
\usepackage{amsmath, amscd} %amscd for commutative diagrams
\usepackage{mathtools}
\usepackage{amssymb}
\usepackage{array}
\usepackage{blkarray}
\usepackage{color}
\usepackage{enumerate}
\usepackage{fancybox}
\usepackage{float}
\usepackage[hang]{footmisc}
\usepackage{graphicx}
\usepackage{listings}
\usepackage{lmodern}
\usepackage{multirow}
\usepackage{pifont}
\usepackage{rotating}
\usepackage[small, sl, SL,sf, SF, bf, hang, raggedright]{subfigure}
\usepackage{tabularx}
\usepackage{tcolorbox}
\usepackage{textpos}
\usepackage{textcomp}
\usepackage{tikz}
\usepackage{upgreek}
\usepackage{appendixnumberbeamer}
\usepackage{courier} % For programming style font

\definecolor{DESYcyan}{RGB}{0,166,235}
\definecolor{DESYorange}{RGB}{242,142,0}
\definecolor{DESYgray}{RGB}{119,119,119}
\definecolor{CBSpurple}{RGB}{117,112,179}
\definecolor{CBSorange}{RGB}{217,95,2}
\definecolor{CBSgreen}{RGB}{27,158,119}
\newcommand\OrangeBFMath[1]{{\color{DESYorange}\boldsymbol{#1}}}
\newcommand\CyanBFMath[1]{{\color{DESYcyan}\boldsymbol{#1}}}

\usepackage[font=footnotesize,labelfont={color=DESYorange,footnotesize}]{caption}

\usepackage{xcolor}
\definecolor{light-gray}{gray}{0.8}
\usepackage{relsize}
\renewcommand*{\UrlFont}{\ttfamily\smaller\relax}
\usepackage{hyperref}
\hypersetup{
	colorlinks=true,
	linkcolor=black,
	filecolor=black,
	urlcolor=black,%light-gray,
	breaklinks=true
}



\usepackage[compat=1.1.0]{tikz-feynman}
\usepackage{contour}

\usetikzlibrary{arrows,shapes,positioning}
\usetikzlibrary{decorations.markings}
\usetikzlibrary{decorations.pathmorphing}
\usetikzlibrary{decorations.pathreplacing}
\usetikzlibrary{patterns}
\usetikzlibrary{plotmarks}
\usetikzlibrary{shadows}

\tcbuselibrary{skins}

\newcommand*\at{{\fontfamily{ptm}\selectfont @}}


%%%%%%%%%%%%%%%%%%%%%%%%%%%%%%%%%%%%%%%%%%%%%%%%%%%%%%%%%%%%%%%%%%%%%%%%%%%%%%%
% Code:

\usepackage{ulem} %for underline, crossing our, etc...

\lstset{
%backgroundcolor=\color{lbcolor},
    tabsize=2,
%   rulecolor=,
    language=[GNU]C++,
    % linewidth=0.9\textwidth,
    xleftmargin=.1\textwidth,
        basicstyle=\scriptsize,
        upquote=true,
        aboveskip={1.5\baselineskip},
        columns=fixed,
        showstringspaces=false,
        extendedchars=false,
        breaklines=true,
        prebreak = \raisebox{0ex}[0ex][0ex]{\ensuremath{\hookleftarrow}},
        frame=single,
        % numbers=left,
        showtabs=false,
        showspaces=false,
        showstringspaces=false,
        identifierstyle=\ttfamily,
        keywordstyle=\color[rgb]{0,0,1},
        commentstyle=\color[rgb]{0.026,0.112,0.095},
        stringstyle=\color[rgb]{0.627,0.126,0.941},
        numberstyle=\color[rgb]{0.205, 0.142, 0.73},
%        \lstdefinestyle{C++}{language=C++,style=numbers}’.
    texcl=true,
}

% \usepackage{filecontent}

\newsavebox\mybox

%%%%%%%%%%%%%%%%%%%%%%%%%%%%%%%%%%%%%%%%%%%%%%%%%%%%%%%%%%%%%%%%%%%%%%%%%%%%%%%


\setlength\footnotemargin{10pt}

\let\beginoldtabular=\tabular
\let\endoldtabular\endtabular
\renewenvironment{tabular}
{
	\def\arraystretch{1.5}
	\beginoldtabular
}
{
	\endoldtabular
}

\newcolumntype{C}{>{\centering\arraybackslash}X}
\let\beginoldtabularx=\tabularx
\let\endoldtabularx\endtabularx
\renewenvironment{tabularx}
{
	\def\arraystretch{1.5}
	\beginoldtabularx
}
{
	\endoldtabularx
}


\usepackage{beamerthemesplit}



% \usetheme{CambridgeUS}
\usefonttheme[onlymath]{serif}
\usecolortheme[named=DESYcyan]{structure}
\useoutertheme{infolines}
% \useoutertheme{default}


\setbeamercolor{palette primary}{bg=white}
\setbeamercolor{section in head/foot}{bg=DESYcyan, fg=white}
\setbeamercolor{frametitle}{fg=DESYcyan,bg=white}
\setbeamerfont{frametitle}{series=\bfseries, size=\LARGE}
\setbeamercolor{framesubtitle}{fg=DESYorange,bg=white}
\setbeamerfont{framesubtitle}{series=\bfseries}
\addtobeamertemplate{headline}{\hypersetup{linkcolor=white}\vspace*{0.7em}}{\vspace*{-0.2em}}
% \setbeamertemplate{headline}{\leavevmode \hypersetup{linkcolor=white} }

\setbeamercolor{item}{fg=DESYorange}
\setbeamercolor{subitem}{fg=DESYcyan}
\setbeamercolor{subsubitem}{fg=DESYgray}

\setbeamertemplate{navigation symbols}{}
\setbeamertemplate{footline}{}
\setbeamertemplate{bibliography item}[text]


\addtobeamertemplate{footline}{}{
\begin{textblock*}{0.9\textwidth}[-0.1,1](-0.5cm, -0.3cm)  %(5cm,5cm)
\textbf{\textcolor{DESYcyan}{DESY}\textcolor{DESYorange}{.}} ~$\quad$ \hfill Page \insertframenumber/\inserttotalframenumber
\end{textblock*}
\begin{textblock*}{0.9\textwidth}[-0.1,1](5cm,5cm)
\end{textblock*}
}

\tcbset{colframe=DESYcyan, colback=DESYcyan!2!white, colupper=black, fonttitle=\bfseries,nobeforeafter,center title}


\setbeamercolor{title}{fg=DESYcyan}
\setbeamerfont{title}{series=\bfseries, size=}
\setbeamercolor{subtitle}{fg=DESYorange}

\newcommand{\myitem}{\item[\ding{228}]}

\newcommand{\BackUp}{\appendix\frame{\begin{center}\begin{Huge}Backup Slides\end{Huge}\end{center}}}


%%%%%%%%%%%%%%%%%%%%%%%%%%%%%%%%%%%%%%%%%%%%%%%%%%%%%%%%%%%%%%%%%%%%%%%%%%%%%%%
% Shadowed Images

\usetikzlibrary{shadows,calc}

% code adapted from https://tex.stackexchange.com/a/11483/3954

% some parameters for customization
\def\shadowshift{3pt,-3pt}
\def\shadowradius{6pt}

\colorlet{innercolor}{black!60}
\colorlet{outercolor}{gray!05}

% this draws a shadow under a rectangle node
\newcommand\drawshadow[1]{
    \begin{pgfonlayer}{shadow}
        \shade[outercolor,inner color=innercolor,outer color=outercolor] ($(#1.south west)+(\shadowshift)+(\shadowradius/2,\shadowradius/2)$) circle (\shadowradius);
        \shade[outercolor,inner color=innercolor,outer color=outercolor] ($(#1.north west)+(\shadowshift)+(\shadowradius/2,-\shadowradius/2)$) circle (\shadowradius);
        \shade[outercolor,inner color=innercolor,outer color=outercolor] ($(#1.south east)+(\shadowshift)+(-\shadowradius/2,\shadowradius/2)$) circle (\shadowradius);
        \shade[outercolor,inner color=innercolor,outer color=outercolor] ($(#1.north east)+(\shadowshift)+(-\shadowradius/2,-\shadowradius/2)$) circle (\shadowradius);
        \shade[top color=innercolor,bottom color=outercolor] ($(#1.south west)+(\shadowshift)+(\shadowradius/2,-\shadowradius/2)$) rectangle ($(#1.south east)+(\shadowshift)+(-\shadowradius/2,\shadowradius/2)$);
        \shade[left color=innercolor,right color=outercolor] ($(#1.south east)+(\shadowshift)+(-\shadowradius/2,\shadowradius/2)$) rectangle ($(#1.north east)+(\shadowshift)+(\shadowradius/2,-\shadowradius/2)$);
        \shade[bottom color=innercolor,top color=outercolor] ($(#1.north west)+(\shadowshift)+(\shadowradius/2,-\shadowradius/2)$) rectangle ($(#1.north east)+(\shadowshift)+(-\shadowradius/2,\shadowradius/2)$);
        \shade[outercolor,right color=innercolor,left color=outercolor] ($(#1.south west)+(\shadowshift)+(-\shadowradius/2,\shadowradius/2)$) rectangle ($(#1.north west)+(\shadowshift)+(\shadowradius/2,-\shadowradius/2)$);
        \filldraw ($(#1.south west)+(\shadowshift)+(\shadowradius/2,\shadowradius/2)$) rectangle ($(#1.north east)+(\shadowshift)-(\shadowradius/2,\shadowradius/2)$);
    \end{pgfonlayer}
}

% create a shadow layer, so that we don't need to worry about overdrawing other things
\pgfdeclarelayer{shadow}
\pgfsetlayers{shadow,main}


\newcommand\shadowimage[2][]{%
\begin{tikzpicture}
\node[anchor=south west,inner sep=0] (image) at (0,0) {\includegraphics[#1]{#2}};
\drawshadow{image}
\end{tikzpicture}}
%%%%%%%%%%%%%%%%%%%%%%%%%%%%%%%%%%%%%%%%%%%%%%%%%%%%%%%%%%%%%%%%%%%%%%%%%%%%%%%
% New symbols


\newcommand{\Lagr}{\mathcal{L}} % Lagrangian-L
\newcommand{\Lumi}{\mathcal{L}} % Lagrangian-L
\newcommand{\oper}{\mathcal{O}} % Operator symbol
\newcommand{\Pol}{\mathcal{P}} % Operator symbol

%%%%%%%%%%%%%%%%%%%%%%%%%%%%%%%%%%%%%%%%%%%%%%%%%%%%%%%%%%%%%%%%%%%%%%%%%%%%%%%
% DESY text colors

\newcommand\OrangeBF[1]{\textcolor{DESYorange}{\textbf{#1}}}
\newcommand\CyanBF[1]{\textcolor{DESYcyan}{\textbf{#1}}}

%%%%%%%%%%%%%%%%%%%%%%%%%%%%%%%%%%%%%%%%%%%%%%%%%%%%%%%%%%%%%%%%%%%%%%%%%%%%%%%
% Convenient way to set border around input

\newcommand\scaledinput[2][]{
  \scalebox{#1}{\input{#2}}
}

%%%%%%%%%%%%%%%%%%%%%%%%%%%%%%%%%%%%%%%%%%%%%%%%%%%%%%%%%%%%%%%%%%%%%%%%%%%%%%%
% Rotated text

\usepackage{rotating}

%%%%%%%%%%%%%%%%%%%%%%%%%%%%%%%%%%%%%%%%%%%%%%%%%%%%%%%%%%%%%%%%%%%%%%%%%%%%%%%
% Particles
\newcommand{\eP}{e^{+}}
\newcommand{\eM}{e^{-}}
\newcommand{\muP}{\mu^{+}}
\newcommand{\muM}{\mu^{-}}
\newcommand{\lP}{l^{+}}
\newcommand{\lM}{l^{-}}

\newcommand{\fbar}{\bar{f}}
\newcommand{\mubar}{\bar{\mu}}

\newcommand{\eL}{\eM_{L}}
\newcommand{\eR}{\eM_{R}}
\newcommand{\pL}{\eP_{L}}
\newcommand{\pR}{\eP_{R}}

\newcommand{\nue}{\nu_e}
\newcommand{\numu}{\nu_{\mu}}
\newcommand{\nutau}{\nu_{\tau}}
\newcommand{\nul}{\nu_l}

\newcommand{\nubar}{\bar{\nu}}
\newcommand{\nuebar}{\bar{\nue}}
\newcommand{\nulbar}{\bar{\nul}}

\newcommand{\lbar}{\bar{l}}

\newcommand{\qbar}{\bar{q}}
\newcommand{\qu}{q_{u}}
\newcommand{\qd}{q_{d}}
\newcommand{\qubar}{\bar{q}_{u}}
\newcommand{\qdbar}{\bar{q}_{d}}
\newcommand{\cbar}{\bar{c}}
\newcommand{\bbar}{\bar{b}}

\newcommand{\tbar}{\bar{t}}

\newcommand{\vis}{\text{vis}}
\newcommand{\inv}{\text{inv}}

\newcommand{\WP}{W^{+}}
\newcommand{\WM}{W^{-}}
%%%%%%%%%%%%%%%%%%%%%%%%%%%%%%%%%%%%%%%%%%%%%%%%%%%%%%%%%%%%%%%%%%%%%%%%%%%%%%%
% Checkmark

\def\checkmark{\tikz\fill[scale=0.4](0,.35) -- (.25,0) -- (1,.7) -- (.25,.15) -- cycle;}
%%%%%%%%%%%%%%%%%%%%%%%%%%%%%%%%%%%%%%%%%%%%%%%%%%%%%%%%%%%%%%%%%%%%%%%%%%%%%%%
% Allow Latex over included graphic
\usepackage[percent]{overpic}

%%%%%%%%%%%%%%%%%%%%%%%%%%%%%%%%%%%%%%%%%%%%%%%%%%%%%%%%%%%%%%%%%%%%%%%%%%%%%%%
% Package for additional symbols like cross-x
\usepackage{pifont}
\newcommand{\xmark}{\ding{55}}%

%%%%%%%%%%%%%%%%%%%%%%%%%%%%%%%%%%%%%%%%%%%%%%%%%%%%%%%%%%%%%%%%%%%%%%%%%%%%%%%
% Allow line-breaks in tabular cell
\usepackage{makecell}

%%%%%%%%%%%%%%%%%%%%%%%%%%%%%%%%%%%%%%%%%%%%%%%%%%%%%%%%%%%%%%%%%%%%%%%%%%%%%%%
% Allow frame around includegraphics
\usepackage[export]{adjustbox}

%%%%%%%%%%%%%%%%%%%%%%%%%%%%%%%%%%%%%%%%%%%%%%%%%%%%%%%%%%%%%%%%%%%%%%%%%%%%%%%
% Transparent memes
\usepackage{transparent}

%%%%%%%%%%%%%%%%%%%%%%%%%%%%%%%%%%%%%%%%%%%%%%%%%%%%%%%%%%%%%%%%%%%%%%%%%%%%%%%
% Define new table column commands that allow a \newline line break
\usepackage{array}
% \newcolumntype{L}[1]{>{\raggedright\let\newline\\\arraybackslash\hspace{0pt}}m{#1}}
% \newcolumntype{C}[1]{>{\centering\let\newline\\\arraybackslash\hspace{0pt}}m{#1}}
% \newcolumntype{R}[1]{>{\raggedleft\let\newline\\\arraybackslash\hspace{0pt}}m{#1}}
%%%%%%%%%%%%%%%%%%%%%%%%%%%%%%%%%%%%%%%%%%%%%%%%%%%%%%%%%%%%%%%%%%%%%%%%%%%%%%%

